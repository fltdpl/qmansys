\documentclass[fleqn]{scrartcl}
\usepackage[automark]{scrpage2}
\usepackage{graphicx}
 
%%% Farben %%%%%%%%%%%%%%%%%%%%%%%%%%%%%%%%%%%%%%%%%%%%%
\usepackage{color}
\definecolor{cgreen}{rgb}{0,0.6,0}
\definecolor{orange}{rgb}{1,0.5,0}
\definecolor{gray05}{gray}{.5}
 
%%% Mathe %%%%%%%%%%%%%%%%%%%%%%%%%%%%%%%%%%%%%%%%%%%%%%
\usepackage{amsmath}
\usepackage{amssymb}
\usepackage{nicefrac}
\usepackage{units}
\usepackage{amsfonts}
\usepackage{textcomp}
 
%%% Code %%%%%%%%%%%%%%%%%%%%%%%%%%%%%%%%%%%%%%%%%%%%%%%
\usepackage{listings}
\lstset{ %
  numbers          = left,
  stepnumber       = 1,
  numbersep        = 5pt,
  numberstyle      = \tiny\color{gray05},
  backgroundcolor   = \color{white},          % background color; add \usepackage{color}
  basicstyle        = \ttfamily\footnotesize, % the size of the fonts that are used for the code
  breakatwhitespace = false,                  % automatic breaks should only happen at whitespace
  showstringspaces  = false,
  breaklines        = true,                   % sets automatic line breaking
  commentstyle      = \color{cgreen},         % comment style
  keywordstyle      = \bfseries\color{blue},
  %identifierstyle  = \color{red},
  stringstyle       = \color{orange},
  language          = octave,
  %frame             = L,
  captionpos        = b,
  }
  % Bsp.: \begin{lstlisting}[caption={bla},label=lst:bla]
  %              ...
  %       \end{lstlisting}
 
%%% Input & Sprache %%%%%%%%%%%%%%%%%%%%%%%%%%%%%%%%%%%%
\usepackage[utf8]{inputenc}
\usepackage[T1]{fontenc}
\usepackage[ngerman]{babel}
 
%%% Geometry %%%%%%%%%%%%%%%%%%%%%%%%%%%%%%%%%%%%%%%%%%%
\usepackage{calc}
\usepackage[paper=a4paper,left=25mm,right=25mm,top=25mm,bottom=25mm]{geometry}
\usepackage{layout}
 
%%% Schrift & Satz %%%%%%%%%%%%%%%%%%%%%%%%%%%%%%%%%%%%%
\usepackage{lmodern}                    % Schriftfamilie lmodern
\addtokomafont{sectioning}{\rmfamily}   % Serifen in Überschriften
\newcommand{\changefont}[3]{
    \fontfamily{#1}\fontseries{#2}\fontshape{#3}\selectfont}
\parindent0mm                           % keine Einrückungen bei neuen Absätzen
\usepackage{caption}                    % Caption Schriftgröße
\captionsetup{
    format=plain,
    font=small,                         % footnotesize
    labelfont=bf,
}
\usepackage{bm}             %fette griechische buchstaben mittels \boldsymbol
\usepackage{upgreek}
%\usepackage[onehalfspacing]{setspace}  % Zeilenabstand
 
%%% Hyperref %%%%%%%%%%%%%%%%%%%%%%%%%%%%%%%%%%%%%%%%%%
\usepackage{hyperref}
\usepackage{url}
\hypersetup{
%  pdftitle     = Titel der Arbeit,
%  pdfsubject   = Thema der Arbeit,
%  pdfauthor    = Autor,
  pdfcreator   = pdflatex,
  pdfproducer  = LaTeX with hyperref,
  colorlinks   = true,
  linkcolor    = black,
  urlcolor     = black,
  citecolor    = black,%green,        % color of links to bibliography
  filecolor    = magenta,      % color of file links
%  bookmarks    = true,         % show bookmarks bar?
%  unicode      = false,        % non-Latin characters in Acrobat’s bookmarks
%  pdftoolbar   = true,         % show Acrobat’s toolbar?
%  pdfmenubar   = true,         % show Acrobat’s menu?
%  pdffitwindow = false,        % window fit to page when opened
  pdfstartview = {FitH}        % fits the width of the page to the window
%  pdfkeywords = {keyword1} {key2} {key3}, % list of keywords
%  pdfnewwindow= true,         % links in new window
}
\usepackage{nameref} % Auf einzelne Kapitel referenzieren

%%% Anhang %%%%%%%%%%%%%%%%%%%%%%%%%%%%%%%%%%%%%%%%%%%
\newcounter{appendix}
\setcounter{appendix}{1}
\renewcommand{\theappendix}{\Alph{appendix}}
\renewcommand{\appendix}[1]{%
\newpage\section*{Anhang\ \theappendix :\ #1}
\addcontentsline{toc}{subsection}{Anhang\ \theappendix :\ #1}
\stepcounter{appendix}
}

 
\begin{document}
 
%%% Hier gehts los %%%
\begin{figure}[htbp]
  \centering
  % GNUPLOT: LaTeX picture with Postscript
\begingroup
  \makeatletter
  \providecommand\color[2][]{%
    \GenericError{(gnuplot) \space\space\space\@spaces}{%
      Package color not loaded in conjunction with
      terminal option `colourtext'%
    }{See the gnuplot documentation for explanation.%
    }{Either use 'blacktext' in gnuplot or load the package
      color.sty in LaTeX.}%
    \renewcommand\color[2][]{}%
  }%
  \providecommand\includegraphics[2][]{%
    \GenericError{(gnuplot) \space\space\space\@spaces}{%
      Package graphicx or graphics not loaded%
    }{See the gnuplot documentation for explanation.%
    }{The gnuplot epslatex terminal needs graphicx.sty or graphics.sty.}%
    \renewcommand\includegraphics[2][]{}%
  }%
  \providecommand\rotatebox[2]{#2}%
  \@ifundefined{ifGPcolor}{%
    \newif\ifGPcolor
    \GPcolorfalse
  }{}%
  \@ifundefined{ifGPblacktext}{%
    \newif\ifGPblacktext
    \GPblacktexttrue
  }{}%
  % define a \g@addto@macro without @ in the name:
  \let\gplgaddtomacro\g@addto@macro
  % define empty templates for all commands taking text:
  \gdef\gplbacktext{}%
  \gdef\gplfronttext{}%
  \makeatother
  \ifGPblacktext
    % no textcolor at all
    \def\colorrgb#1{}%
    \def\colorgray#1{}%
  \else
    % gray or color?
    \ifGPcolor
      \def\colorrgb#1{\color[rgb]{#1}}%
      \def\colorgray#1{\color[gray]{#1}}%
      \expandafter\def\csname LTw\endcsname{\color{white}}%
      \expandafter\def\csname LTb\endcsname{\color{black}}%
      \expandafter\def\csname LTa\endcsname{\color{black}}%
      \expandafter\def\csname LT0\endcsname{\color[rgb]{1,0,0}}%
      \expandafter\def\csname LT1\endcsname{\color[rgb]{0,1,0}}%
      \expandafter\def\csname LT2\endcsname{\color[rgb]{0,0,1}}%
      \expandafter\def\csname LT3\endcsname{\color[rgb]{1,0,1}}%
      \expandafter\def\csname LT4\endcsname{\color[rgb]{0,1,1}}%
      \expandafter\def\csname LT5\endcsname{\color[rgb]{1,1,0}}%
      \expandafter\def\csname LT6\endcsname{\color[rgb]{0,0,0}}%
      \expandafter\def\csname LT7\endcsname{\color[rgb]{1,0.3,0}}%
      \expandafter\def\csname LT8\endcsname{\color[rgb]{0.5,0.5,0.5}}%
    \else
      % gray
      \def\colorrgb#1{\color{black}}%
      \def\colorgray#1{\color[gray]{#1}}%
      \expandafter\def\csname LTw\endcsname{\color{white}}%
      \expandafter\def\csname LTb\endcsname{\color{black}}%
      \expandafter\def\csname LTa\endcsname{\color{black}}%
      \expandafter\def\csname LT0\endcsname{\color{black}}%
      \expandafter\def\csname LT1\endcsname{\color{black}}%
      \expandafter\def\csname LT2\endcsname{\color{black}}%
      \expandafter\def\csname LT3\endcsname{\color{black}}%
      \expandafter\def\csname LT4\endcsname{\color{black}}%
      \expandafter\def\csname LT5\endcsname{\color{black}}%
      \expandafter\def\csname LT6\endcsname{\color{black}}%
      \expandafter\def\csname LT7\endcsname{\color{black}}%
      \expandafter\def\csname LT8\endcsname{\color{black}}%
    \fi
  \fi
  \setlength{\unitlength}{0.0500bp}%
  \begin{picture}(5760.00,4320.00)%
    \gplgaddtomacro\gplbacktext{%
      \colorrgb{0.00,0.00,0.00}%
      \put(620,640){\makebox(0,0)[r]{\strut{}0}}%
      \colorrgb{0.00,0.00,0.00}%
      \put(620,1131){\makebox(0,0)[r]{\strut{}10}}%
      \colorrgb{0.00,0.00,0.00}%
      \put(620,1623){\makebox(0,0)[r]{\strut{}20}}%
      \colorrgb{0.00,0.00,0.00}%
      \put(620,2114){\makebox(0,0)[r]{\strut{}30}}%
      \colorrgb{0.00,0.00,0.00}%
      \put(620,2605){\makebox(0,0)[r]{\strut{}40}}%
      \colorrgb{0.00,0.00,0.00}%
      \put(620,3096){\makebox(0,0)[r]{\strut{}50}}%
      \colorrgb{0.00,0.00,0.00}%
      \put(620,3588){\makebox(0,0)[r]{\strut{}60}}%
      \colorrgb{0.00,0.00,0.00}%
      \put(620,4079){\makebox(0,0)[r]{\strut{}70}}%
      \colorrgb{0.00,0.00,0.00}%
      \put(740,440){\makebox(0,0){\strut{}0}}%
      \colorrgb{0.00,0.00,0.00}%
      \put(1322,440){\makebox(0,0){\strut{}10}}%
      \colorrgb{0.00,0.00,0.00}%
      \put(1905,440){\makebox(0,0){\strut{}20}}%
      \colorrgb{0.00,0.00,0.00}%
      \put(2487,440){\makebox(0,0){\strut{}30}}%
      \colorrgb{0.00,0.00,0.00}%
      \put(3070,440){\makebox(0,0){\strut{}40}}%
      \colorrgb{0.00,0.00,0.00}%
      \put(3652,440){\makebox(0,0){\strut{}50}}%
      \colorrgb{0.00,0.00,0.00}%
      \put(4234,440){\makebox(0,0){\strut{}60}}%
      \colorrgb{0.00,0.00,0.00}%
      \put(4817,440){\makebox(0,0){\strut{}70}}%
      \colorrgb{0.00,0.00,0.00}%
      \put(5399,440){\makebox(0,0){\strut{}80}}%
      \colorrgb{0.00,0.00,0.00}%
      \put(160,2359){\rotatebox{90}{\makebox(0,0){\strut{}Mischtemperatur \textdegree C}}}%
      \colorrgb{0.00,0.00,0.00}%
      \put(3069,140){\makebox(0,0){\strut{}Temperatur des Boilerwassers in \textdegree C}}%
    }%
    \gplgaddtomacro\gplfronttext{%
      \csname LTb\endcsname%
      \put(4496,1003){\makebox(0,0)[r]{\footnotesize Umgebungstemperatur: 0\textdegree C}}%
      \csname LTb\endcsname%
      \put(4496,803){\makebox(0,0)[r]{\footnotesize Umgebungstemperatur: 40\textdegree C}}%
    }%
    \gplbacktext
    \put(0,0){\includegraphics{Tempberechnung}}%
    \gplfronttext
  \end{picture}%
\endgroup

\end{figure}
\end{document}
